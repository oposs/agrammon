\section{Input Parameters}
\tablefirsthead{\hline\bf Parameter & \bf Unit & \bf Description\\\hline}
\tablehead{\hline\bf Parameter & \bf Unit & \bf Description\\}
\tabletail{\hline}
\begin{xtabular}{|p{5cm}|l|p{0.5\textwidth}|}
\multicolumn{3}{@{}l}{}\\
\multicolumn{3}{@{}l}{\footnotesize\bf Livestock::Poultry::Excretion}\\\hline
animalcategory & - & Poultry category (layers, growers, broilers, turkeys, and other poultry).\begin{flushleft}Possible~values:  \texttt{ broilers}, \texttt{ growers}, \texttt{ other\_poultry}, \texttt{ turkeys}, \texttt{layers}\end{flushleft} \\\hline
animals & - & Number of poultry animals for the selected type in barn. \\\hline
\multicolumn{3}{@{}l}{}\\
\multicolumn{3}{@{}l}{\footnotesize\bf Livestock::Poultry::Outdoor}\\\hline
free\-\_range & - & Average free range hours per day.\begin{flushleft}Possible~values:  \texttt{ no}, \texttt{yes}\end{flushleft} \\\hline
\multicolumn{3}{@{}l}{}\\
\multicolumn{3}{@{}l}{\footnotesize\bf Livestock::Poultry::Housing::Type}\\\hline
housing\-\_type & - & Type of housing.\begin{flushleft}Possible~values:  \texttt{ deep\_litter}, \texttt{ deep\_pit}, \texttt{manure\_belt}\end{flushleft} \\\hline
manure\-\_removal\-\_interval & - & Manure removal interval by manure belt.\begin{flushleft}Possible~values:  \texttt{ 3\_to\_4\_times\_a\_month}, \texttt{ more\_than\_4\_times\_a\_month}, \texttt{ no\_manure\_belt}, \texttt{ twice\_a\_month}, \texttt{less\_than\_twice\_a\_month}\end{flushleft} \\\hline
drinking\-\_system & - & Type of drinking system.\begin{flushleft}Possible~values:  \texttt{ bell\_drinkers}, \texttt{drinking\_nipples}\end{flushleft} \\\hline
\multicolumn{3}{@{}l}{}\\
\multicolumn{3}{@{}l}{\footnotesize\bf Livestock::Poultry::Housing::AirScrubber}\\\hline
air\-\_scrubber & - & Exhaust air scrubber: none, acid, biotrickling\_filter.\begin{flushleft}Possible~values:  \texttt{ acid}, \texttt{ biotrickling}, \texttt{none}\end{flushleft} \\\hline
\multicolumn{3}{@{}l}{}\\
\multicolumn{3}{@{}l}{\footnotesize\bf Storage::SolidManure::Poultry}\\\hline
share\-\_applied\-\_direct\-\_poultry\-\_manure & \% & Share of poultry manure applied to land without storage. \\\hline
share\-\_covered\-\_basin & \% & Share of droppings or mist from poultry stored in covered basin. \\\hline
\multicolumn{3}{@{}l}{}\\
\multicolumn{3}{@{}l}{\footnotesize\bf Storage::SolidManure::Solid}\\\hline
share\-\_applied\-\_direct\-\_cattle\-\_other\-\_manure & \% & Share of cattle manure applied to land without storage. \\\hline
share\-\_applied\-\_direct\-\_pig\-\_manure & \% & Share of pig manure applied to land without storage. \\\hline
\multicolumn{3}{@{}l}{}\\
\multicolumn{3}{@{}l}{\footnotesize\bf Application::Slurry::Ctech}\\\hline
share\-\_splash\-\_plate & \% & Share of slurry applied with splash plate. \\\hline
share\-\_trailing\-\_hose & \% & Share of slurry applied with trailing hose. \\\hline
share\-\_trailing\-\_shoe & \% & Share of slurry applied with trailing shoes. \\\hline
share\-\_shallow\-\_injection & \% & Share of slurry applied with shallow injection. \\\hline
share\-\_deep\-\_injection & \% & Share of slurry applied with deep injection. \\\hline
\multicolumn{3}{@{}l}{}\\
\multicolumn{3}{@{}l}{\footnotesize\bf Application::Slurry::Applrate}\\\hline
dilution\-\_parts\-\_water & 1:x & Specific slurry dilution. TAN contents have been calculated based on a
   standard dilution of 1:1 with a TAN content of 1.15 kg N /m3.   \\\hline
appl\-\_rate & m3 /ha & Application rate, mean volume of slurry applied on a ha per deployment. \\\hline
\multicolumn{3}{@{}l}{}\\
\multicolumn{3}{@{}l}{\footnotesize\bf Application::Slurry::Csoft}\\\hline
appl\-\_evening & \% & Share of slurry applied in the evening after 18:00. \\\hline
appl\-\_hotdays & - & Proportion of slurry applied on hot days.\begin{flushleft}Possible~values:  \texttt{ never}, \texttt{ rarely}, \texttt{ sometimes}, \texttt{frequently}\end{flushleft} \\\hline
\multicolumn{3}{@{}l}{}\\
\multicolumn{3}{@{}l}{\footnotesize\bf Application::Slurry::Cseason}\\\hline
appl\-\_summer & \% & Share of slurry applied June to August (in \%). \\\hline
appl\-\_autumn\-\_winter\-\_spring & \% & Share of slurry applied September to May. \\\hline
\multicolumn{3}{@{}l}{}\\
\multicolumn{3}{@{}l}{\footnotesize\bf Application::SolidManure::CincorpTime}\\\hline
incorp\-\_lw1h & \% & Share of incorporated solid manure within 1 hour.

 \\\hline
incorp\-\_lw4h & \% & Share of incorporated solid manure within 4 hours.

 \\\hline
incorp\-\_lw8h & \% & Share of incorporated solid manure within 8 hours.

 \\\hline
incorp\-\_lw1d & \% & Share of incorporated solid manure within 1 day.

 \\\hline
incorp\-\_lw3d & \% & Share of incorporated solid manure within 3 days.

 \\\hline
incorp\-\_gt3d & \% & Share of incorporated solid manure after 3 days.

 \\\hline
incorp\-\_none & \% & Share of solid manure not incorporated.

 \\\hline
\multicolumn{3}{@{}l}{}\\
\multicolumn{3}{@{}l}{\footnotesize\bf Application::SolidManure::Cseason}\\\hline
appl\-\_summer & \% & Share of solid manure applied June to August (in \%). \\\hline
appl\-\_autumn\-\_winter\-\_spring & \% & Share of solid manure applied September to May (in \%). \\\hline
\multicolumn{3}{@{}l}{}\\
\multicolumn{3}{@{}l}{\footnotesize\bf PlantProduction::AgriculturalArea}\\\hline
agricultural\-\_area & ha & Agricultural area.
 \\\hline
\multicolumn{3}{@{}l}{}\\
\multicolumn{3}{@{}l}{\footnotesize\bf PlantProduction::MineralFertiliser}\\\hline
mineral\-\_nitrogen\-\_fertiliser\-\_urea & kg N /a & Amount of urea in kg N /a.
 \\\hline
mineral\-\_nitrogen\-\_fertiliser\-\_except\-\_urea & kg N /a & Amount of nitrogen fertiliser (except urea) in kg N /a.
 \\\hline
\multicolumn{3}{@{}l}{}\\
\multicolumn{3}{@{}l}{\footnotesize\bf PlantProduction::RecyclingFertiliser}\\\hline
compost & t /a & Amount of compost (in t fresh matter per year).
    Kompost besteht aus Grünabfällen nicht-landwirtschaflticher Herkunft von gewerblich-industriellen Anlagen oder von Feldrandkompostierung.
 \\\hline
solid\-\_digestate & t /a & Amount of solid digestate form industrial factories.
 \\\hline
liquid\-\_digestate & m3 /a & Amount of liquid digestate form industrial factories.
 \\\hline
\end{xtabular}
\section{Technical Parameters}
\tablefirsthead{\hline\bf Parameter & \bf Value & \bf Unit & \bf Description\\\hline}
\tablehead{\hline\bf Parameter & \bf Value & \bf Unit & \bf Description\\}
\tabletail{\hline}
\begin{xtabular}{|p{5cm}|l|l|p{0.5\textwidth}|}
\multicolumn{4}{@{}l}{}\\
\multicolumn{4}{@{}l}{\footnotesize\bf Livestock::Poultry::Excretion}\\\hline
standard\-\_N\-\_excretion\-\_layers & 0.80 & kg N /a & Annual standard N excretion for poultry category (layers) according to 
    Flisch et al. (2009). \\\hline
standard\-\_N\-\_excretion\-\_growers & 0.34 & kg N /a & Annual standard N excretion for poultry category (growers) according to
    Flisch et al. (2009). \\\hline
standard\-\_N\-\_excretion\-\_broilers & 0.45 & kg N /a & Annual standard N excretion for poultry category (broilers) according to
    Flisch et al. (2009). \\\hline
standard\-\_N\-\_excretion\-\_turkeys & 1.4 & kg N /a & Annual standard N excretion for poultry category according (turkeys) to
    Flisch et al. (2009). \\\hline
standard\-\_N\-\_excretion\-\_other\-\_poultry & 0.56 & kg N /a & Annual standard N excretion for other poultry category according to
    Flisch et al. (2009). \\\hline
share\-\_Nsol\-\_layers & 0.6 & - & Nsol content of excreta for layers. Derived from e.g.
    TODO \\\hline
share\-\_Nsol\-\_growers & 0.6 & - & Nsol content of excreta for growers. Derived from e.g.
    TODO \\\hline
share\-\_Nsol\-\_broilers & 0.6 & - & Nsol content of excreta for broilers. Derived from e.g.
    TODO \\\hline
share\-\_Nsol\-\_turkeys & 0.6 & - & Nsol content of excreta for turkeys. Derived from e.g.
    TODO \\\hline
share\-\_Nsol\-\_other\-\_poultry & 0.6 & - & Nsol content of excreta for other poultry. Derived from e.g.
    TODO \\\hline
\multicolumn{4}{@{}l}{}\\
\multicolumn{4}{@{}l}{\footnotesize\bf Livestock::Poultry::Outdoor}\\\hline
er\-\_free\-\_range & 0.7 & - & Emission rate for free range poultry, based on Menzi et al. (1997): 70\% of TAN or 28\% of Ntot \\\hline
free\-\_range\-\_days\-\_layers & 280 & d /a & Average free range days per year. \\\hline
free\-\_range\-\_hours\-\_layers & 2.88 & h /d & Average free range hours per day, assumed is 12\% of Day \\\hline
free\-\_range\-\_days\-\_growers & 280 & d /a & Average free range days per year. \\\hline
free\-\_range\-\_hours\-\_growers & 2.88 & h /d & Average free range hours per day, assumed is 12\% of Day \\\hline
free\-\_range\-\_days\-\_turkeys & 280 & d /a & Average free range days per year. \\\hline
free\-\_range\-\_hours\-\_turkeys & 0.96 & h /d & Average free range hours per day, assumed is 4\% of Day \\\hline
free\-\_range\-\_days\-\_other\-\_poultry & 280 & d /a & Average free range days per year. \\\hline
free\-\_range\-\_hours\-\_other\-\_poultry & 0.96 & h /d & Average free range hours per day, assumed is 12\% of Day \\\hline
free\-\_range\-\_days\-\_broilers & 280 & d /a & Average free range days per year. \\\hline
free\-\_range\-\_hours\-\_broilers & 0.96 & h /d & Average free range hours per day, assumed is 4\% of Day \\\hline
\multicolumn{4}{@{}l}{}\\
\multicolumn{4}{@{}l}{\footnotesize\bf Livestock::Poultry::Housing::Type}\\\hline
er\-\_housing\-\_layers\-\_growers\-\_manure\-\_belt & 0.15 & - & Emission rate for the poultry housing type, based on EAGER workshop January 2007: 15\% of Ntot, converted using 60\% Nsol and emission factor of 25\%. \\\hline
er\-\_housing\-\_layers\-\_growers\-\_deep\-\_pit & 0.30 & - & Emission rate for the poultry housing type, based on EAGER workshop January 2007, UNECE 2007: 30\% of Ntot, converted using 60\% Nsol and emission factor of 50\%. \\\hline
er\-\_housing\-\_layers\-\_growers\-\_deep\-\_litter & 0.30 & - & Emission rate for the poultry housing type, based on EAGER workshop January 2007, UNECE 2007: 30\% of Ntot, converted using 60\% Nsol and emission factor of 50\%. \\\hline
er\-\_housing\-\_other\-\_deep\-\_litter & 0.12 & - & Emission rate for the poultry housing type, based on Reidy et al. (2009): 12\% of Ntot, converted using 60\% Nsol and emission factor of 20\%. \\\hline
c\-\_manure\-\_removal\-\_interval\-\_less\-\_than\-\_twice\-\_a\-\_month & 1.2 & - & Emission rate for the poultry manure removal by droppings belt. Empirical assumption by Reidy/Menzi.  \\\hline
c\-\_manure\-\_removal\-\_interval\-\_twice\-\_a\-\_month & 1 & - & Emission rate for the poultry manure removal by droppings belt. Empirical assumption by Reidy/Menzi.  \\\hline
c\-\_manure\-\_removal\-\_interval\-\_3\-\_to\-\_4\-\_times\-\_a\-\_month & 0.8 & - & Emission rate for the poultry manure removal by droppings belt. Empirical assumption by Reidy/Menzi.  \\\hline
c\-\_manure\-\_removal\-\_interval\-\_more\-\_than\-\_4\-\_times\-\_a\-\_month & 0.6 & - & Emission rate for the poultry manure removal by droppings belt. Empirical assumption by Reidy/Menzi.  \\\hline
c\-\_drinking\-\_nipples & 1.0 & - & Emission rate for the poultry drinking type standard version. \\\hline
c\-\_bell\-\_drinkers & 1.2 & - & Emission rate for the poultry drinking type additional emission. Empirical assumption by Reidy/Menzi. 
  \\TODO: Give better description! \\\hline
\multicolumn{4}{@{}l}{}\\
\multicolumn{4}{@{}l}{\footnotesize\bf Livestock::Poultry::Housing::AirScrubber}\\\hline
red\-\_acid\-\_air\-\_scrubber & 0.9 & - & Reduction efficiency as compared to group-housed on fully and partly slatted floors (UNECE 2007, paragraph 71, table 5). \\\hline
red\-\_biotrickling\-\_filter\-\_air\-\_scrubber & 0.7 & - & Reduction efficiency as compared to group-housed on fully and partly slatted floors (UNECE 2007, paragraph 71, table 5). \\\hline
\multicolumn{4}{@{}l}{}\\
\multicolumn{4}{@{}l}{\footnotesize\bf Storage}\\\hline
mineralizationrate\-\_liquid & 0.1 & - & A netto mineralization of 10\% from Norg to NSol/TAN is assuemd, according to
  the GAS\_EM Model \\\hline
\multicolumn{4}{@{}l}{}\\
\multicolumn{4}{@{}l}{\footnotesize\bf Storage::SolidManure::Poultry}\\\hline
er\-\_layers\-\_growers\-\_other\-\_poultry & 0.25 & - & Emission rate for layers, growers and other poultry for manure (deep pit, deep litter) and droppings (manure belt)(based on EAGER workshop, January 2008: 15\% Ntot, converted using Nsol 60\% and emission factor of 25\%. \\\hline
er\-\_turkeys\-\_broilers & 0.1 & - & Emission rate for  manure of broilers and turkeys based on EAGER workshop, January 2008: 6\% Ntot, converted using Nsol 60\% and emission factor of 10\%. \\\hline
c\-\_droppings\-\_mist\-\_covered\-\_basin & 0.6 & - & Reduction of emission rate for the droppings or mist stored in covered basin for poultry. \\\hline
immobilizationrate\-\_poultry & 0.4 & - & A netto immobilization of 40\% from NSol/TAN to Norg is assuemd, according to
  the GAS\_EM Model \\\hline
\multicolumn{4}{@{}l}{}\\
\multicolumn{4}{@{}l}{\footnotesize\bf Storage::SolidManure::Solid}\\\hline
er\-\_tan\-\_pigs & 0.5 & - & The value has been derived from the Eager workshop, January 2008: (additional explanation following) \\\hline
er\-\_tan\-\_cattle\-\_other & 0.3 & - & The value has been derived from the Eager workshop, January 2008: (additional explanation following) \\\hline
immobilizationrate\-\_solid & 0.4 & - & A netto immobilization of 40\% from NSol/TAN to Norg is assuemd, according to
  the GAS\_EM Model \\\hline
\multicolumn{4}{@{}l}{}\\
\multicolumn{4}{@{}l}{\footnotesize\bf Application::Slurry}\\\hline
er\-\_App\-\_cattle\-\_liquid & 0.5 & - & Emission rate for slurry application based on TAN of the slurry. The 
    average rate has been derived from Sommer (2001b), Sogaard et al. (2002), Menzi et al. (1998), Menzi et al. (1997a) \\\hline
er\-\_App\-\_pigs\-\_liquid & 0.35 & - & Die Emissionsrate wurde gemäss ALFAM Modell (Sogaard et al., 2002) berechnet mit folgenden Inputdaten: durchschnittliche Temperatur von März bis November: 12°C (Daten SMA Station Bern Liebefeld 1993-2002); Windgeschwindigkeit von 1 m/s:
Schweinegülle Mast: TAN Gehalt Gülle: 2.1 kg/m3 (Verdünnung 1:1, d.h. 2.5 \% TS gemäss Flisch et al., 2009); ohne Korrekturen für emissionsminderende Ausbringung, ohne Einarbeitung nach Ausbringung; Ausbringungsmenge: 30 m3/ha; mikrometeorologische Messung: 30.3 \% TAN (Mittelwert Boden feucht, Boden trocken). Bei gleichen Annahmen, jedoch einer reduzierten Ausbringungsmenge von 20 m3/ha (aufgrund des im Vergleich zu Rindergülle höheren TAN-Gehalts) und eines TS Gehalts von 3 \% (höherer Strohanteil bei Labelsystemen): 33.2 \%.
Unter den analogen Annahmen resultieren für Schweinegülle Zucht (TAN Gehalt Gülle: 1.65 kg/m3; Verdünnung 1:1, d.h. 2.5 \% TS gemäss Flisch et al., 2009) Emissionsraten von 32.9 \% bzw. 36.2 \% TAN. \\\hline
\multicolumn{4}{@{}l}{}\\
\multicolumn{4}{@{}l}{\footnotesize\bf Application::Slurry::Ctech}\\\hline
red\-\_splash\-\_plate & 0.0 & - & There is no reduction for broadcasting with splash plate as to this way of 
    applying slurry all the other methods are compared to. \\\hline
red\-\_trailing\-\_hose & -0.3 & - & Reduction efficiency as compared to broadcasting applying trailing hose.
    Adopted from UNECE (2007), Frick and Menzi (1997) 
    and Menzi et al. (1997). \\\hline
red\-\_trailing\-\_shoe & -0.5 & - & Reduction efficiency as compared to broadcasting applying trailing shoe. 
    Adopted from UNECE (2007), Frick and Menzi (1997) 
    and Menzi et al. (1997). \\\hline
red\-\_shallow\-\_injection & -0.7 & - & Reduction efficiency as compared to broadcasting applying shallow injection.
    Adopted from UNECE (2007), Frick and Menzi (1997) 
    and Menzi et al. (1997). \\\hline
red\-\_deep\-\_injection & -0.8 & - & Reduction efficiency as compared to broadcasting applying deep injection.
    Adopted from UNECE(2007), Frick and Menzi (1997) 
    and Menzi et al. (1997).  \\\hline
\multicolumn{4}{@{}l}{}\\
\multicolumn{4}{@{}l}{\footnotesize\bf Application::Slurry::Applrate}\\\hline
norm\-\_er & 0.5 & - & Standard emission of 50\% of the applied TAN calculated based on an
    equation published by Menzi et al (1998) using a TAN standard of 1.15 
    kg /m3 for an 1:1 dilution, with application rate (AR) standard of 30 m3 /ha and average 
    swiss meteorological conditions ( T=12 C, humitity=70\%):
    ((19.41 * TAN\_standard + 4.2 * 1.102 - 9.51) * (0.0214 * ARstandard + 0.36) / (AR\_standard * TAN-standard))) \\\hline
\multicolumn{4}{@{}l}{}\\
\multicolumn{4}{@{}l}{\footnotesize\bf Application::Slurry::Csoft}\\\hline
c\-\_evening & -0.2 & - & Correction factor of the emission rate if slurry is applied in the
    evening (after 18h)(Menzi et al 1997; Frick and Menzi 1997).

    Assumption based on a single experiment with an application after 18h in August at a temperature of >20°C: reduction of the emission by 38\%, the reduction of the emission averaged over the whole year is only 50\%, i.e. -0.2
The correction is omitted for solid manure since infiltration into soil does not occur. \\\hline
c\-\_hotdays\-\_frequently & 0.1 & - & Correction factor of the emission rate if slurry is applied frequently 
    on hot days.

    Loss calculated according to the model of Katz (Menzi et al. 1997b) at 17°C (i.e. +5°C) compared to the reference temperature of 12°C (other parameters: 70\% rela-tive air humidity, 1.15 kg/m3 TAN, 30 m3/ha) resulting in a loss of 19.22 kg N/ha at 17 °C and 55.7\% TAN, respectively (compared to 17.45 kg N/ha and 50.6\% TAN at 12°C, respectively) which corresponds to an increase of 10.1\% (rounded to 10\%). \\\hline
c\-\_hotdays\-\_sometimes & 0.0 & - & Correction factor of the emission rate if slurry is applied sometimes
    on hot days (estimation based on Menzi et al (1997)). \\\hline
c\-\_hotdays\-\_rarely & -0.1 & - & Correction factor of the emission rate if slurry is applied rarely
    on hot days (estimation based on Menzi et al (1997)). \\\hline
c\-\_hotdays\-\_never & -0.2 & - & Correction factor of the emission rate if slurry is applied never
    on hot days (estimation based on Menzi et al (1997)). \\\hline
\multicolumn{4}{@{}l}{}\\
\multicolumn{4}{@{}l}{\footnotesize\bf Application::Slurry::Cseason}\\\hline
c\-\_summer & 0.15 & - & Correction factor for the application of slurry in summer (June to
  August): Model calculation according to the model of Katz (Menzi et
  al. 1997b) with meteorological data from Liebefeld 1993-2002:
  average from March to November 12°C, 70\% relative air humidity, 1.15
  kg/m3 TAN, 30 m3/ha resulting in a loss of 50.6\% TAN; summer 17.8°C
  resulting in a loss of 56.7\% TAN (+12\%). Value chosen for
  cal-culation: +15\% \\\hline
c\-\_autumn\-\_winter\-\_spring & -0.05 & - & Correction factor for the application of slurry in autumn, winter and spring (Sept to May): Model calculation according to the model of Katz (Menzi et
  al. 1997b) with meteorological data from Liebefeld 1993-2002: average
  from March to November 12°C, 70\% relative air humidity, 1.15 kg/m3
  TAN, 30 m3/ha resulting in a loss of 50.6\% TAN; spring/autumn/winter 9°C
  resulting in a loss of 48.1\% TAN (-4.8\%). Value chosen for
  calculation: -5\% \\\hline
\multicolumn{4}{@{}l}{}\\
\multicolumn{4}{@{}l}{\footnotesize\bf Application::SolidManure::Solid}\\\hline
er\-\_App\-\_manure\-\_dairycows\-\_cattle\-\_pigs & 0.8 & - & Emission rate for manure application. The average rate has been
  derived from Frick et al. (1996) and Menzi et al. (1996). The value is
  based on the average emissions from diffrent Swiss
  experiments. Emission based on TAN of slurry. \\\hline
er\-\_App\-\_manure\-\_horses\-\_otherequides\-\_smallruminants & 0.7 & - & Emission rate for manure application. The average rate has been
  derived from Frick et al. (1996) and Menzi et al. (1996). The value is
  based on the average emissions from diffrent Swiss
  experiments. Emission based on TAN of slurry. \\\hline
\multicolumn{4}{@{}l}{}\\
\multicolumn{4}{@{}l}{\footnotesize\bf Application::SolidManure::Solid::CincorpTime}\\\hline
eff\-\_inc\-\_lw1h & -0.9 & - & Reduction due to incorporation of solid manure within 1 hour.
    UNECE (2007). 
 \\\hline
eff\-\_inc\-\_lw4h & -0.7 & - & Reduction due to incorporation of solid manure within 4 hours.
     Empirical estimate deduced from UNECE (2007).
     Mean value between the category incorporation within 1 hour and incorporation within 8 hours.
 \\\hline
eff\-\_inc\-\_lw8h & -0.5 & - & Reduction due to incorporation of solid manure within 8 hours.
     Values adapted from UNECE (2007) (category Incorporation by plough within 12 h)
 \\\hline
eff\-\_inc\-\_lw1d & -0.35 & - & Reduction due to incorporation of solid manure within 1 day.
    Values adapted from UNECE (2007)
    Empirical estimate deduced from Menzi et al. (1997).
 \\\hline
eff\-\_inc\-\_lw3d & -0.3 & - & Reduction due to incorporation of solid manure within 3 days.
    Empirical estimate deduced from Menzi et al. (1997).
 \\\hline
eff\-\_inc\-\_gt3d & -0.1 & - & Reduction due to incorporation of solid manure after 3 days
    Empirical estimate deduced from Menzi et al. (1997).
 \\\hline
eff\-\_inc\-\_none & 0.0 & - & Basis with no incorporation of solid manure.
 \\\hline
\multicolumn{4}{@{}l}{}\\
\multicolumn{4}{@{}l}{\footnotesize\bf Application::SolidManure::Cseason}\\\hline
c\-\_summer & 0.15 & - & Correction factor for the application of solid manure in summer (June to August):
Model calculation according to the model of Katz (Menzi et al. 1997b) with meteorological data from Liebefeld 1993-2002: average from March to November 12°C, 70\% relative air humidity, 1.15 kg/m3 TAN, 30 m3/ha resulting in a loss of 50.6\% TAN; summer 17.8°C resulting in a loss of 56.7\% TAN (+12\%). Value chosen for cal-culation: +15\%. \\\hline
c\-\_autumn\-\_winter\-\_spring & -0.05 & - & Correction factor for the application of solid manure in autumn, winter and spring (Sept to May):
  Model calculation according to the model of Katz (Menzi et al. 1997b) with meteorological data from Liebefeld 1993-2002: average from March to November 12°C, 70\% relative air humidity, 1.15 kg/m3 TAN, 30 m3/ha resulting in a loss of 50.6\% TAN; spring/autumn/winter 9°C resulting in a loss of 48.1\% TAN (-4.8\%). Value chosen for calculation: -5\%. \\\hline
\multicolumn{4}{@{}l}{}\\
\multicolumn{4}{@{}l}{\footnotesize\bf Application::SolidManure::Poultry}\\\hline
er\-\_App\-\_manure\-\_layers\-\_growers\-\_other\-\_poultry & 0.3 & - & Emission rate for manure application. The average rate has been
  derived from Frick et al. (1996) and Menzi et al. (1997). The value is
  based on the average emissions from diffrent Swiss
  experiments. Emission based on TAN of slurry. \\\hline
er\-\_App\-\_manure\-\_turkeys\-\_broilers & 0.65 & - & Emission rate for manure application. The average rate has been
  derived from Frick et al. (1996) and Menzi et al. (1997). The value is
  based on the average emissions from diffrent Swiss
  experiments. Emission based on TAN of slurry. \\\hline
\multicolumn{4}{@{}l}{}\\
\multicolumn{4}{@{}l}{\footnotesize\bf Application::SolidManure::Poultry::CincorpTime}\\\hline
eff\-\_inc\-\_lw1h & -0.95 & - & Reduction due to incorporation of solid manure within 1 hour.
    UNECE (2007). 
 \\\hline
eff\-\_inc\-\_lw4h & -0.8 & - & Reduction due to incorporation of solid manure within 4 hours.
     Empirical estimate deduced from UNECE (2007).
     Mean value between the category incorporation within 1 hour and incorporation within 8 hours.
 \\\hline
eff\-\_inc\-\_lw8h & -0.7 & - & Reduction due to incorporation of solid manure within 8 hours.
     Values adapted from UNECE (2007) (category Incorporation by plough within 12 h)
 \\\hline
eff\-\_inc\-\_lw1d & -0.55 & - & Reduction due to incorporation of solid manure within 1 day.
    Values adapted from UNECE (2007)
    Empirical estimate deduced from Menzi et al. (1997).
 \\\hline
eff\-\_inc\-\_lw3d & -0.3 & - & Reduction due to incorporation of solid manure within 3 days.
    Empirical estimate deduced from Menzi et al. (1997).
 \\\hline
eff\-\_inc\-\_gt3d & -0.1 & - & Reduction due to incorporation of solid manure after 3 days
    Empirical estimate deduced from Menzi et al. (1997).
 \\\hline
eff\-\_inc\-\_none & 0.0 & - & Basis with no incorporation of solid manure.
 \\\hline
\multicolumn{4}{@{}l}{}\\
\multicolumn{4}{@{}l}{\footnotesize\bf PlantProduction::AgriculturalArea}\\\hline
er\-\_agricultural\-\_area & 2 & kg N /ha /a & Emission rate from the agricultural area. The average rate has been
  derived from Schjoerring and Mattson (2001). Emission based on kg/ ha AA (AA = agricultural area, Landwirschaftliche Nutzfläche). N ist NH3 N.
 \\\hline
\multicolumn{4}{@{}l}{}\\
\multicolumn{4}{@{}l}{\footnotesize\bf PlantProduction::MineralFertiliser}\\\hline
er\-\_App\-\_mineral\-\_nitrogen\-\_fertiliser\-\_urea & 0.15 & - & Emission rate for the application of urea. The average rate has been
  derived from Vanderweerden and Jarvis (1997). Emission based on Ntot.
 \\\hline
er\-\_App\-\_mineral\-\_nitrogen\-\_fertiliser\-\_except\-\_urea & 0.02 & - & Emission rate for the application of ammonium nitrate. The average rate has been
  derived from Vanderweerden and Jarvis (1997). Emission based on Ntot.
 \\\hline
\multicolumn{4}{@{}l}{}\\
\multicolumn{4}{@{}l}{\footnotesize\bf PlantProduction::RecyclingFertiliser}\\\hline
er\-\_compost & 0.24 & kg N / t & Emission rate from compost, calculated with an emmission rate of 80 \\% TAN and a fraction of
  0.3 kg TAN per t fresh matter (Flisch et al., 2009).
  of TAN.
 \\\hline
er\-\_solid\-\_digestate & 0.24 & kg N / t & Emission rate for solid digestat from industrial plantse, calculated with an emmission rate
  of 80 \\% TAN and a fraction of  0.3 kg TAN per t fresh matter (Flisch et al., 2009).
 \\\hline
er\-\_liquid\-\_digestate & 1.2 & kg N / t & Emission rate from liquid digestate from industrial plants, calculated with an emmission
  rate of 60 \\% TAN and a fraction of 2 kg TAN per t fresh matter (Flisch et al., 2009).
 \\\hline
\end{xtabular}
